\documentclass[11pt,dark]{mathbeamer}
\usepackage{xecjk}
\usepackage{fontspec}
\usepackage{cite}
\usepackage[author-year]{amsrefs}
\setCJKmainfont{SimSun}
\setmainfont{SimSun}
\title{关于比特币期权的实证研究 }
\subtitle{本科生综合论文训练 开题报告}
\author[代港凡 2015012502]{代港凡 2015012502 \and 指导教师:王茵田}
\institute{清华大学经济管理学院}
\date{\today}
\setbeamertemplate{footline}
{
  \leavevmode%
  \hbox{%
  \begin{beamercolorbox}[wd=.3\paperwidth,ht=2.25ex,dp=1ex,center]{author in head/foot}%
    \usebeamerfont{author in head/foot}\insertshortauthor
  \end{beamercolorbox}%
  \begin{beamercolorbox}[wd=.4\paperwidth,ht=2.25ex,dp=1ex,center]{title in head/foot}%
    \usebeamerfont{title in head/foot}\insertshorttitle
  \end{beamercolorbox}%
  \begin{beamercolorbox}[wd=.3\paperwidth,ht=2.25ex,dp=1ex,center]{date in head/foot}%
    \insertframenumber{} / \inserttotalframenumber\hspace*{1ex}
  \end{beamercolorbox}}%
  \vskip0pt%
}
\begin{document}
\section*{目录}
\begin{frame}{目录}
    \tableofcontents
\end{frame}
\section{选题描述}
\frame{\tableofcontents[currentsection]}
\subsection{选题背景}
\begin{frame}{选题背景}
    2008年,中本聪在网络上发表了题为“比特币:一种点对点的电子货币系统”\cite{Nakamoto_bitcoin:a},正式提出了第一种完善的加密货币。
    因为其去中心化、安全、私密的特点,比特币很快受到了大众欢迎。虽然学术界对于比特币是否是货币、是否有价值仍存在较大的争议,但不可否认的是比特币已经成为了一种金融资产。
\end{frame}
\begin{frame}{选题背景}
    随着比特币交易市场的渐渐成熟,人们对于投机交易和风险控制的需求也逐渐提升。各大交易所纷纷推出了比特币期货或者远期合约。在此基础上,Deribit和LedgerX先后于2016年和2017年推出了比特币期权合约。
    \newline 这些期权合约均为欧式,条款简单,但同时也为从另一个维度研究比特币市场提供了手段。
    \newline 本篇论文旨在利用已有的比特币交易数据,通过对期权真实价格和理论价格之间差异的研究,发现比特币市场摩擦的来源,同时论证利用合理模型进行对冲交易的可行性。
  \end{frame}
\subsection{选题目的}
\begin{frame}{选题目的}
  
\end{frame}
\section{Reference}
    \begin{frame}[t, allowframebreaks]{Reference}
        \bibliographystyle{plain}
		\bibliography{slides/bib}
	\end{frame}
\end{document}
