\chapter{数据}
\section{数据来源}
\subsection{比特币数据}
\par{
    
    本文主要采用期权交易所LedgerX(https://www.ledgerx.com)公开的每日交易数据,数据位于其网站的“数据”栏目上(data.ledgerx.com),该交易所的所有期权均为欧式期权。数据包括期权合约条款(行权价、是否为认购、到期日)、每日成交量加权平均价、成交量、持仓量、最后一次买价和卖价等。其中一天的数据示例如下图所示:
    \begin{figure}[H]
        \begin{small}
            \begin{center}
                \includegraphics[width=0.95\textwidth]{figures/data_example.png}
            \end{center}
            \caption{LedgerX数据展示}
            \label{data_example}
        \end{small}
    \end{figure}
    利用python爬虫获取了2017年10月17日至2018年12月31日的全部数据,剔除了如图\ref{data_example}中所示的隔夜互换、交易量为0(只有买价或卖价)的数据之后,得到共计1726条期权的交易数据。然而根据参考文献\cite{J-1972}\cite{Jame-1979},一般认为离到期日较近的期权、价值更虚的期权往往定价不准确,因此之后进行回归时会做进一步筛选。期权数据的成交量加权平均价可被认为是当天该期权的平均成交价格水平,故后文提到的“市场价格”均指该期权的成交量加权平均价。
}
\subsection{比特币数据}
\par{
    比特币有众多交易所,根据前述参考文献可知,不同交易所之间价格存在一定差异,且用不同正常货币交易的交易所之间套利空间更为巨大\cite{Makarov-2018},考虑到期权行权和交易均以美元报价,这里我们限制选择以美元为交易货币的交易所,根据比特币数据汇总网站Bitcoinity(https://bitcoinity.org)提供数据,可得到所有美元交易所中,每日结算价中最高者和最低者之比,这一比值走势如下图:
    \begin{figure}[H]
        \begin{small}
            \begin{center}
                \includegraphics[width=0.95\textwidth]{figures/maxmin_ratio_plot.png}
            \end{center}
            \caption{交易所间最高价与最低价之比}
            \label{maxmin_ratio}
        \end{small}
    \end{figure}
    }
    \par{从图中可见交易所之间的价格差异在18年初较大,为保证之后回归的准确和套利策略的可行性,我们选择coinmarketcap网站(https://coinmarketcap.com)上公开的各美元交易所的比特币加权均价,这样可以保证以交易量较大的比特币交易所价格为基础,作为套利策略应用价格较为可行。
    }
\subsection{比特币无风险利率信息}
    \par{各种期权定价模型均需要有无风险利率信息,而比特币市场上并无公开的无风险利率。根据2015年出版的《比特币手册(Bitcoin Handbook)》\cite{WESNER2015223},作者提出了通过宏观经济理论推断比特币无风险利率的方法,根据其提供的结果,可得比特币的无风险利率在4$\%$到6$\%$之间,故取比特币无风险利率为5$\%$。
    }

\section{数据描述}
    \par{
        对数据搜集期比特币每日收盘价数据取对数收益率,按照一年滚动窗口计算其波动率、偏度、峰度。对其描述性统计如下:}

        \begin{threeparttable}[H]
            
            \centering
            \caption{比特币数据描述性统计}
            \label{btc_describe}
            \begin{tabular}{lrrrrr}
\toprule
{} &             成交量 &   对数收益率 &     波动率 &      偏度 &      峰度 \\
\midrule
count &         545.000 & 545.000 & 545.000 & 545.000 & 545.000 \\
mean  &  6756320922.978 &  -0.000 &   0.047 &  -0.097 &   2.669 \\
std   &  3790698285.665 &   0.044 &   0.007 &   0.188 &   0.804 \\
min   &  1403920000.000 &  -0.185 &   0.032 &  -0.541 &   1.656 \\
25\%   &  4273640000.000 &  -0.017 &   0.043 &  -0.313 &   2.135 \\
50\%   &  5475579904.000 &   0.002 &   0.049 &  -0.042 &   2.327 \\
75\%   &  7826525254.000 &   0.018 &   0.054 &   0.060 &   3.041 \\
max   & 23840899072.000 &   0.225 &   0.055 &   0.175 &   5.026 \\
\bottomrule
\end{tabular}

            \begin{tablenotes}
                \footnotesize
                \item 注:波动率、偏度、峰度用一年(365天)滚动窗口估计,均为日化数据;偏度为超额偏度;成交量以美元计量;数据保留三位小数。
            \end{tablenotes}
        \end{threeparttable}
        
    
    ~\\
    \par{
        
        由表\ref{btc_describe}可见,数据期比特币存在一定的负偏度与高峰度。其中峰度一直保持在0以上,说明比特币收益率存在偏峰厚尾现象,而偏度数据相对波动较大,总体上比较接近0偏度,同时对数收益率均值也非常接近0,说明比特币的收益分布非常接近存在尖峰厚尾现象的对称分布。
    }
    \newpage
    \par{
        将期权交易数据按照其标签(即合约的到期日、行权价、认购/认沽方向)分组,并对基本信息统计如下:
    }
    \par{
    \begin{threeparttable}[H]
        \centering
        \caption{期权数据描述性统计}
        \label{option_describe}
        \begin{tabular}{lrrrr}
\toprule
{} &      期限 &       行权价 &  认购期权数量 &  认沽期权数量 \\
\midrule
数目  & 306.000 &   306.000 & 197.000 & 109.000 \\
均值  &  90.536 &  8867.647 &         &         \\
标准差 & 141.132 &  6207.192 &         &         \\
最小值 &   1.000 &  2000.000 &         &         \\
中位数 &  30.000 &  7000.000 &         &         \\
最大值 & 637.000 & 50000.000 &         &         \\
\bottomrule
\end{tabular}

        \begin{tablenotes}
            \footnotesize
            \item 注:此处统计已将期权按照到期日、行权价和方向分组。期限为第一个有交易的日期至到期日的天数。
        \end{tablenotes}
    \end{threeparttable}
    }
    ~\\
    \par{
    从表\ref{option_describe}可见,认购期权数量略多于认沽期权,行权价的覆盖范围较广。存在一部分期限极短的数据,可能存在之前无交易的情况,会在实证研究中考虑将这类数据删去。
    }