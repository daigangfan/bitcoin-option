\thusetup{
    ctitle={关于比特币期权的实证研究},
    cdegree={经济学学士},
    cdepartment={经济管理学院},
    cmajor={经济与金融},
    cauthor={代港凡},
    csupervisor={王茵田}    
}
\begin{cabstract}
    本文利用LedgerX等比特币期权交易所的公开数据,对比特币期权的特性进行了研究。主要分为以下三个步骤:利用经典的Black-Scholes模型对比特币期权进行定价,并得到模型价格与真实价格之间差异;选取比特币交易量、收益率、峰度、非流动性指标等市场因素与期权价值程度、持仓量、到期时间、隐含波动率等自身特性构建回归模型对价格差异进行解释;利用模型价格和真实价格之间的差异构建套利策略,考察比特币市场的有效性以及Black-Scholes模型蕴含的信息价值。本文主要得到了Black-Scholes模型对大量比特币期权的定价结果,同时发现比特币的流动性、期权的时间、隐含波动率溢价等信息对期权的价格差异均有系统性影响,并从期权供需、套利的可能角度做出了解释。同时也发现对于本文用到的套利策略,认购期权相比认沽期权有更稳定的套利收益。
\end{cabstract}
\ckeywords{比特币, 期权, B-S模型,套利}
\begin{eabstract}
    This research makes use of open data from LedgerX, a bitcoin option exchange, to take a research on some characteristics of bitcoin options. There are mainly three parts in this paper: first I use famous Black-Scholes model to price the options, and get a suitable definition of the difference between real trading price and model(theoretical) price; second I build regression model to explain the difference using some variables: market variables like bitcoin returns, kurtosis, etc., as well as option variables like implied volatility, time to maturity, etc.; third I develop an arbitrage strategy between market price and model price. This research gets pricing results for options in a long time range, and identifies some systematic relationship between pricing difference and variables like bitcoin liquidity, implied volatility premium, etc.. As for the arbitrage strategy developed in this research, call options gains more stable and significant returns compared with puts.
\end{eabstract}
\ekeywords{bitcoin, option, arbitrage}