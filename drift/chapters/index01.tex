\chapter{外文资料的调研阅读报告或书面翻译}
\title{期权合约价值与市场有效性检验}
\section{引言}
\par{期权合约指一种在特定时间内以给定价格买卖另一种资产的权力。购买普通股的权证,执行官的股票期权,认购和认沽期权都是期权合约的常见案例。认沽和认购期权是本文主要讨论的话题。看涨期权合约是在合约到期之前的任何时间(到期日)以固定价格(行权价)购买100股证券的权利。认沽合约是以合约到期日以行权价出售100股证券的权利。其他常见合约是看跌期权和看涨期权的组合。例如,跨式合约是一个看跌期权和一个看涨期权的组合。
当前期权市场的组织结构由Boness描述,并在芝加哥期货交易所最近的一项研究中有详细描述。}

\par{期权合约的有效期通常以月为单位,以便合约价格不受期权到期后发生的意外事件的影响。 对于看跌期权或看涨期权的每个买方,都有对应的卖方,因此行使期权不会导致公司普通股的数量增加。 行权价格会针对股票股息和拆分进行调整,并根据普通股支付的股息金额进行调整。}
\par{
在最近的一篇论文中,布莱克和斯科尔斯得出了一个期权理论价值的公式。 本文的目的是测试该估值模型,并将从模型中得出的期权价值与期权合约的实际费用进行比较。 由于该模型使用基础证券价格的历史波动率来进行估计,市场交易者可能能够通过使用与过去数据估计的方差相关的其他信息来战胜估值模型。 我们计划阐明现在期权是如何定价的并测试期权交易商在确定这些价格时的有效性。 我们还将记录当前期权市场的交易成本,以证明在目前的期权市场组织中不存在盈利机会。 由于期权是许多金融合约不可或缺的一部分,我们希望验证我们模型的实证有效性,作为更复杂的其他期权形式模型的先行者。}
\section{理论定价模型}
\par{
    在推导看涨期权的定价模型时,我们假设(1)普通股的收益波动率在期权合约的有效期内是恒定的,并且对市场参与者已知; (2)短期利率在合约有效期内是已知且不变的,而该利率是市场参与者的借入和借出利率; (3)期权持
    有人不受影响股价的派息的影响; (4)在有限的时间间隔内,普通股的收益率
    是对数正态分布的。这些假设需要与一般均衡模型中通常的假设相结合,即没
    有交易成本,交易者使用证券卖空所得收益的能力,以及可以利用短期利率的
    无限借贷。这些假设是否过于严格以至于无法用于分析,只能通过对衍生模型
    进行实证检验来解决。然而,这些假设比其他一般均衡模型的限制更小,因为我
    们不需要就证券的预期回报具有一致预期,也不需要就其联合概率回报分布的
    特征达成一致,以及证券是否是公平定价的。
    在这些假设下期权合约价值仅是普通股价格和时间的函数。我们发现在期
    权和股票达到均衡状态蒋存在一个唯一的期权价值。如果期权的价格与模型给
    出的价格不同,则期权中的多(空)头和相应普通股中的空(多)头组合在一
    起,实际上在短期没有风险,同时获得高于短期利率的回报。此对冲组合中期权
    和股票的比例必须随着时间的推移进行调整,以考虑股票价格和合约期限的变
    化。
    由于期权的价值可能会暂时偏离其均衡价值,因此这种对冲在短期内可能
    无法盈利。但是期权必须最终回到其均衡价值,因为它的价值最大值为零,股票
    价格与到期日的价格之间的差价也是如此。由于处于均衡状态,无风险对冲无
    法产生大于市场短期利率的回报,因此必须对期权进行定价,使市场参与者无
    法建立此对冲并期望实现可靠的利润。
    由于期权的价值可能会暂时偏离其均衡价值,因此这种对冲在短期内可能
    无法盈利。但是期权必须最终回到其均衡价值,因为在到期日,它的价值为0 和
    股价与行权价之间的差价。由于处于均衡状态,无风险对冲无法产生大于市场
    短期利率的回报,因此必须期权的定价必须位于使市场参与者无法建立此对冲
    并期望实现可靠的利润水平。
    阻止套利策略获利的价值公式为:
}
\begin{equation}
    w=xN(d_1)-ce^(-rt*)N(d_2)
\end{equation}
\par{
    其中,w 为对于一股的期权价格,x 为当前股票价格,c 为期权行权价,r 为
短期利率,$t*$为期权期限,$d_1=\frac{ln(x/c)+(r+1/2\sigma^2)t*}{\sigma*{\sqrt{t*}}}$,$d_2=d_1-\sigma*\sqrt{t*}$,
N(d)为正态分布
的累计概率密度函数,$\sigma^2$ 为股票收益的波动率。
}
\par{
    期权的价值是利率r 和方差,$\sigma^2$ 的函数,而不是股票预期收益的函数。由于看跌期权和看涨期权的持续时间较短,因此可以假设利率不变或通过使用与
    合约期限相匹配的短期市场工具来估计利率。波动率会产生更严重的问题。我
    们将使用根据历史数据估算的方差来测试定价模型。
    假设模型和市场交易者都尽可能地使用历史方差估计中包含的信息,但市
    场交易者使用该模型没有的未来波动率的其他信息来源。还假设我们将模型价
    格与市场价格进行比较,并假设我们购买看似“低估”的合约并出售看似“高
    估”的合约。如果交易以模型价格进行,我们很可能赔钱,因为市场有模型没有
    的信息。如果交易以市场价格进行,我们很可能既不会赚很多钱,也不会损失大
    笔资金,因为该模型没有市场没有的任何信息。
    另一方面,假设市场拥有模型没有的信息,但未能正确使用历史波动率估
    计中包含的信息。然后,如果交易以模型价格进行,我们可能会赔钱,但如果交
    易以市场价格进行,则可以赚钱。市场有效性的测试将是确定我们是否可以通
    过购买“低估”期权以及以市场价格出售“高估”期权来赚取大量利润。在下一
    节中,我们将描述我们的样本,用于测试估值模型和市场定价的方法,以及我们
    的分析结果。在下一节中,我们将记录当前期权市场的交易成本。
}
\section{定价模型的实证检验}
\par{
    我们获取了1966-1969 年期权经纪人的日志。期权经纪人记录了为其客户
构建的所有期权合约。这些日志包含行权价,到期日,出售日期,收到的期权费,
实际行权日期以及书面合同数量。从日志中我们记录了纽约证券交易所证券的
6 个月认购期权和6 个月跨式期权的数据。我们使用ISL 数据带来获得出售期权
的545 种证券中的每一种的每日收盘价,股息和资本变化。仔细筛选数据以保
持一致性。筛选后,我们的样本包含2,039 份看涨期权合约和3,052 份跨式合约。
让我们将w(x,t*)定义为给定当前股票价格x 和期权的持续时间t *,以
行权价c 购买一股股票的期权的价值。Black 和Scholes 证明,如果我们平衡每
个期权与w1(x,t *)\footnote{w1为w的一阶导数}股的股票,我们会在每个时刻创建一个近似无风险的对冲
组合。
}
\par{
    Black 和Scholes 证明,对每个期权需要的对冲股票数量,$w_1(x,t*)$,为:
}
\begin{equation}
    w_1(x,t^*)=N(d_1)
\end{equation}
\par{
    根据式(1) 中的描述:
}
\begin{equation}
    d_1=\frac{ln(x/c)+(r+1/2\sigma^2)t*}{\sigma*{\sqrt{t*}}}
\end{equation}
\par{
    $N(d_1)$可以取值的范围为从0 到1,我们可以看到每个期权对应的股票数
量将随着股票价格x 和合约期限t* 的变化而变化。当t* 变为零时,$d_1$将接近
负或正无穷大,具体取决于股票价格是低于还是高于行权价格。因此,在合约到
期时,每个期权对应股票数量将为零或一。此外,如果股票价格上涨,我们用更
大比例的股票平衡我们的期权,如果股票价格下跌,我们将用更小比例的股票
平衡我们的期权。
}
\par{
    每天我们通过使用公式(1)计算w 的估计值,给出我们对普通股的收益方
差的估计。每天我们使用公式(2)计算$w_1xt*$ 以在普通股中建立我们的套期头
寸。作为对短期利率r 的估计,我们使用了合约出售月份中的六个月商业票据利
率。这六个月的汇率被转换为连续复利。为了估计方差,我们使用了普通股在合
约出售日期回溯一年的日回报波动率。
}
\par{在此分析中,我们假设我们在购买和出售证券时不会产生任何交易成本,并
且我们可以通过以证券的每日收盘价买入或卖出股票来调整我们在股票上的对
冲头寸。我们假设期权卖方收到的期权费也是期权买家支付的期权费,因为我
们的数据源仅包含卖方收到的保费。本节中的市场价格是卖方收到的价格,正
如我们在下一节中看到交易成本的影响时,期权买家支付的价格要高得多。
}
\par{
    作为评估定价模型和市场交易者表现的第一步,我们计算了合约未行权的每个交易日样本中每个套期保值的实现超额现金回报。 实现的超额现金收益定义为:
}
\begin{equation}
    \delta{w}-w_1\delta{x}-[w-w_1x]\delta{t}
\end{equation}
\par{
    其中,$\delta{w}$为交易日间期权模型价值变化,$w_1$为每个期权需要的股票数量,$\delta{x}$交易日间股价变化,x为股票价格,w为期权m模型价值,r为利率,$\delta{t}$期权价值变化和股票价格变化计算的时间间隔。
}
\par{
    这些实现的超额美元回报每天加总,形成一个总投资组合的超额现金回报。构建投资组合是为了模拟买入或卖出期权的实际过程,并在期权和证券中建立对冲组合。总的来说,每日投资组合超额收益是在1966年5月至1969年7月之间的766个交易日内计算出来的。在期权卖出日期,将模型用于定价。如果看涨期权的市场价格高于模型价格,我们称之为“高估”合约。如果市场价格低于模型价格,我们称之为“低估”合约。我们基于几种策略构建了四个投资组合。在第一个投资组合策略中,我们以模型价格购买了所有看涨期权合约。在第二个投资组合策略中,我们以市场价格购买了所有看涨期权合约。在第三个投资组合策略中,我们买入了“被低估”的看涨期权并以模型价格出售了“高估”的看涨期权。在第四个投资组合策略中,我们买入了“被低估”的看涨期权并以市场价格出售了“高估”的看涨期权。对于我们每个投资组合策略
计算每日投资组合超额现金回报。
}
\par{
    为了直观地解释这些投资组合策略,我们可以设想以模型价格然后以市场价格购买期权。 以模型价格购买所有期权将测试定价模型是否给出平均来说太高或太低的价格。 以市场价格购买所有期权将测试卖方是否收到平均过高或过低的期权费。 购买“低估”合约并以模型价格出售“高估”合约将测试该模型是否可用于定价合约。 购买“低估”合约并以市场价格出售“高估”合约将测试期权市场在样本期内是否存在套利机会。
}
\par{
    如前所述,模型价格是通过使用基于过去数据的方差估计得出的。 如果期权交易者拥有影响期权价值的未来波动率的其他信息,那么购买“被低估”的合约并以模型价格出售“高估”合约将导致显著负的超额投资组合回报。 如果市场交易者拥有该模型正在使用的所有信息并且他们正确使用这些信息,那么购买“被低估”的合约并以市场价格出售“高估”合约将导致超额投资组合回报与零无显着差异。 如果模型使用的是市场未使用的信息,那么如果我们以市场价格买卖合约,那么超额回报将大大超过零。
}
\par{
    正如Black,Scholes[1]所示,超额现金回报在理论上与市场投资组合的回报无关。 我们通过用标准普尔指数$\tilde{R}_{m,t}$在全部时间段以及十个子时间段上\footnote{
        投资组合收益以美元计量,而指数收益以比例计量,尽管这一差异影响了斜率参数的量级但并不影响我们关心的显著性。估计得到的斜率参数以美元为单位。
    }回归超额现金收益$\tilde{R}_{p,t}$来测试这一点。 使用的回归模型是:
}
\begin{equation}
    \tilde{R}_{p,t}=\alpha_p+\beta_p\tilde{R}_{m,t}+\tilde{\epsilon_t}
\end{equation}
\par{
    其中$\alpha_p$和$\beta_p$是回归参数,$\epsilon_t$是同期每日的残差。
}
\par{
    我们将使用$\hat{\alpha}_p$即估计的截距及其显著性作为衡量模型和市场投资组合各自表现。 估计的斜率系数$\hat{\beta}_p$与0无显著差异,这与预期一致。因此我们仅提供估计的截距,它们的t统计量,以及估计的序列相关性$\rho$。将各种组合的结果列在表1中。
}
\par{
    在表1的A组中,我们按每个合约总结了结果。也就是说,在运行回归之前,我们将每日超额总现金投资组合回报除以每天持仓的合约数量。在表格的B组中,我们总结了以现金总回报为基础的结果。在每组的前两个部分,我们总结了以模型价格购买每个合同的结果,然后是市场价格。在每组的最后两部分中,我们总结了购买“被低估”合约并以模型价格然后以市场价格出售“高估”合约的结果。
}
\par{
    当我们以模型价格购买合约时,超额投资组合的回报与零无显著差异\footnote{
        由于投资组合收益似乎确实存在显着的序列相关性,我们也计算了月回报,并重新运行等式(4)以获得截距的估计及其对各种投资组合的t-统计量。 然后,我们将$\hat{\alpha}$除以每月的天数以与日频结果可比。 作为另一个检验,我们对子周期$\hat{\alpha}$进行平均并计算这些子周期的平均值的t统计量。 我们将使用子周期均值的t统计量作为显著性的度量,因为这是为了校正序列相关性问题的尝试,月度结果和子周期的平均$\hat{\alpha}$如表1所示。  每个子期间对于每个合约运行大约需要77天,而对总体的现金运行大约需要61天,因为我们排除了在总现金运行中少于100个持仓合约的初始阶段。
    }。在某些时期,我们实现了显著的正回报,而在其他时期,我们实现了显著的负回报。当我们以市场价格购买合约时,全部时间结果表明投资组合的回报是微不足道的。然而,除了期权卖方收到的价格太低的前两个子期间(正$\hat{\alpha}$),最后八个子期间平均显示出显著的负回报。在这些最后的子期间,卖方收到的合约价格过高。由于合同数量在样本期间增加,因此随着时间的推移,投资组合收益的方差可能存在异方差性。这可能会导致我们低估结果的显著性。
}
\par{
    当我们购买“低估”合约并以模型价格出售“高估”合约时,超额投资组合回报显着为负。平均 $\hat{\alpha}$是每个合约每天-56美元,对应总现金收益为-141.40美元。但是,市场并未使用过去波动率的所有可用信息。如果我们按照市场价格买卖,我们每份合约每天赚0.56美元,按总现金收益每天赚110.69美元。在每个子时段中,结果基本相同。如果我们以模型价格买卖,我们会损失很多,如果我们以市场价格进行买卖,我们会赚很多。
}
\par{
    有必要澄清一下,这些实证研究结果不是通过使用估计的方差来建立不正确套期保值的结果。不正确的套期保值会导致投资组合超额收益的差异增加,但不会影响任何可预测方向的平均超额收益。不正确的套期保值与适当的套期保值加上股票的头寸相同。但由于我们是做多合约以及股票中相反头寸的空头合约,我们在股票中的净头寸可能会随时间平均化为零。
}
\par{这些结果的一个可能的解释是,如果我们知道在期权期间适用的方差,我们计算得到的期权价格往往落在我们实际计算的模型价格和市场价格之间。使用过去数据计算的方差受到测量误差的影响,因此估计方差的分布中的价差大于真实方差中的实际价差。因此,使用有噪声的方差的模型将倾向于高估高方差股票的期权而低估低方差股票的期权。然而,期权交易者对方差分布中差异的估计似乎过于狭窄。高方差证券的期权价格往往过低,低方差证券期权的价格往往过高。期权交易者似乎利用他们过去的经验来得出平均期权价格,但意识到存在方差的变化并且产生了围绕均值分散开来的价格估计。但是,这种分散并不够远。}
\par{
    如果期权的真实价值倾向于介于模型和市场价格之间,那么使用该模型我们将倾向于以过高的价格购买高方差证券的期权,并以过低的价格出售低差异证券的期权(以模型价格买卖),因为市场交易者系统地低估了高方差证券的期权价值,并系统地高估了低方差证券的期权价值。 为了检验这个假设,我们构建了仅基于方差形成的投资组合的超额收益。我们将所有股票从最小方差排列到最大方差,并将估计方差最小的股票的25%分配给第一个投资组合,接下来的25%分配给第二个投资组合,依此类推。我们以模型价格和市场价格购买了每个投资组合的合约。每个投资组合的超额投资组合回报均转换为以每个合约基础,并根据标准普尔指数的回报进行回归。总结结果如表2所示。
}
\par{
    我们从这些模型中清楚地看到,该模型高估了高方差证券期权的价值(我们每个合约每天损失0.35美元和0.36美元)并且低估了低方差证券的期权价值(我们每天每期权分别盈利0.15美元和0.06美元)。 对于市场价格则恰恰相反。 市场交易者低估了高方差证券期权的价值,并高估了低方差证券期权的价值。 如果没有交易成本,在此样本期间期权市场似乎确实存在盈利机会。
}
\par{为了证明如果模型具有适当的方差,就可以正确定价,我们计算了在合约期限内普通股的实际回报差异,并使用此差异对合约定价。每个投资组合如前所述使用基于实际方差率的期权值来确定合约是否被“低估”或“高估”。这些结果总结在表3中。}
\par{
当我们以模型或市场价格购买合约时,我们获得的结果大致与表1中所示相同。但是,与表1中的结果相反,当我们购买“被低估”的合约并卖出“高估”时按模型价格计算合约,我们意识到投资组合的超额回报与零无显著差异。当然,如果我们知道期权的正确价格,并且可以以市场价格买卖合约,我们可以赚取大量资金,但这是对市场定价的一种有严重偏差测试。}
\par{
    当我们使用在期权期间应用的实际方差来对冲合约时,超额投资组合回报的方差要低得多。此外,当我们使用实际方差并以模型价格买入和卖出时,投资组合回报中的序列相关性远小于表1中所示的相关性。投资组合中序列相关性这一发现令人费解。这可能是基础普通股回报中的序列相关性或在我们形成投资组合时引发的。方差中的误差导致我们在股票中占据头寸,如果基础股票收益中存在序列相关性,这将导致投资组合收益也是序列相关的。此外,即使每个证券的回报率是连续不相关的,也可能会引入投资组合收益的序列相关性,因为报价的收盘价可能不是实际的日终价格。我们将对序列相关的问题以及异方差性在未来的工作中进行可能的修正。

}
