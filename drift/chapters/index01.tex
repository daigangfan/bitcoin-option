\chapter{外文资料的调研阅读报告或书面翻译}
\title{期权合约价值与市场有效性检验}
\section{引言}
\par{期权合约指一种在特定时间内以给定价格买卖另一种资产的权力。购买普通股的权证,执行官的股票期权,认购和认沽期权都是期权合约的常见案例。认沽和认购期权是本文主要讨论的话题。看涨期权合约是在合约到期之前的任何时间(到期日)以固定价格(行权价)购买100股证券的权利。认沽合约是以合约到期日以行权价出售100股证券的权利。其他常见合约是看跌期权和看涨期权的组合。例如,跨式合约是一个看跌期权和一个看涨期权的组合。
当前期权市场的组织结构由Boness描述,并在芝加哥期货交易所最近的一项研究中有详细描述。}

\par{期权合约的有效期通常以月为单位,以便合约价格不受期权到期后发生的意外事件的影响。 对于看跌期权或看涨期权的每个买方,都有对应的卖方,因此行使期权不会导致公司普通股的数量增加。 行权价格会针对股票股息和拆分进行调整,并根据普通股支付的股息金额进行调整。}
\par{
在最近的一篇论文中,布莱克和斯科尔斯得出了一个期权理论价值的公式。 本文的目的是测试该估值模型,并将从模型中得出的期权价值与期权合约的实际费用进行比较。 由于该模型使用基础证券价格的历史波动率来进行估计,市场交易者可能能够通过使用与过去数据估计的方差相关的其他信息来战胜估值模型。 我们计划阐明现在期权是如何定价的并测试期权交易商在确定这些价格时的有效性。 我们还将记录当前期权市场的交易成本,以证明在目前的期权市场组织中不存在盈利机会。 由于期权是许多金融合约不可或缺的一部分,我们希望验证我们模型的实证有效性,作为更复杂的其他期权形式模型的先行者。}