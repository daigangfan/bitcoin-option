\chapter{相关研究综述}
\section{数字货币市场相关研究}
\par{数字货币市场在近年来不断升温之后也受到了学术研究的关注。Liu(2018)的文章研究了三种数字货币(比特币、Ripple和以太币)上的风险与收益关系\cite{NBERw24877}。他首先借鉴股票市场上的因子模型,探究了数字货币收益率在CAPM模型、三因子、四因子、五因子模型中的暴露度,发现数字货币的收益率在这些模型中均有显著的alpha,同时对于除了HML(高账面市值比-低账面市值比)之外的因子均无显著的暴露。这说明数字货币的收益与股票市场无明显同步性和相关性。同时对于货币市场因子(如加元、澳元等外汇收益率)、贵金属商品市场因子(如金、铂金、银市场收益率)和宏观因子(如消费增长率、个人收入增长率等)也无明显暴露,并且有稳定而显著的alpha。这说明比特币和其他传统金融市场几乎无同步性。对于特定于数字货币市场上的因子,检验发现极短时间段内的动量效应非常显著,同时投资者注意力,如谷歌搜索指数、推特指数对未来几天比特币收益有显著正向影响,而比特币黑客攻击对未来比特币收益有显著负向影响。市场的波动率对未来收益的影响则并不显著。这篇研究利用股票市场风险模型研究数字货币市场的特征,为我们部分变量的构建提供了思路。}
\par{Makarov等人于2018年发表的文章探究了比特币和其他数字货币市场上的价格形成与效率问题\cite{Makarov-2018}。他们首先计算了全球众多交易所成交量加权均价中最大者和最小者之比,构建了套利指数,随后发现这一指数显著大于一,证明存在明显的套利空间。而对于各个不同的地理区域内部构建的套利指数,要明显更加接近1。因此他们构建了跨区域的套利策略,将一个地区价格较低的交易量在另一个价格较高的地区卖出,并证明国际间的套利有较高的理论收益。随后他们建立了订单流模型,通过分析价格和交易量之间的关系发现,各个交易所的交易量的异质性波动是驱动价格偏离共同水平的重要因素。他们认为套利空间主要来自于对各国对真实货币资本跨国流通的管制,这导致国家间的比特币市场流动性不佳,同时,投资者对套利策略实现的意愿不强也来自于政府管制等特殊风险,说明了套利策略意愿也是对市场有效性比较重要的影响因素。这一研究证明了比特币市场的价格发现机制相对仍不完善,以及流动性对于比特币价格发现的重要意义,为我们套利策略的构建提供了佐证。}

\section{流动性相关研究}
\par{流动性是影响市场有效性的重要因素之一。Shiller(1984)最早提出了市场上存在着两类投资者——精明投资者和噪声投资者\cite{Rober-1984}。当市场的套利成本为0时,精明投资者能够充分利用市场信息套利,使价格回到有效市场水平;随着套利成本增加,噪声投资者对股价的预期的影响会逐渐增加,导致价格偏离有效市场水平。套利成本涉及交易成本、价格滑点等多种与流动性有关的因素。因此我会加入衡量市场流动性的指标来解释期权真实价格与模型价格之间的差异。我主要选择的指标是Amihud(2002)提出的非流动性指标\cite{Yako-2002}。这一指标通过衡量股价收益率对成交量的敏感性来度量市场的流动性,为目前在全球市场上比较通用的指标,具体公式可见第\ref{research_method}章。}
\section{期权定价领域研究}
\par{
期权定价问题一直是金融研究中的热点。Black和Scholes(1972)提出了经典的Black-Scholes模型\cite{10.2307/1831029},其中包含了一系列关于市场有效性的假设,并且对这一模型在市场上的应用进行了实证检验\cite{J-1972}。他们采用滚动窗口的股价收益波动率方法,对纽交所上部分股票的期权进行定价,同时构建对冲策略,证实利用模型信息买入定价低于市场价格、卖出定价高于市场价格的期权同时卖出和买入对应股票的策略可以获得正向的超额收益。同时通过对不同波动率组别的收益进行分析,发现了对于具有较高波动率的股票,定价模型会高估期权的价格。同时他们也验证了套利策略中包含的交易成本是非常巨大的,可能会阻碍套利的进行。}
\par{
之后有众多研究对Black-Scholes模型进行检验或者作出改进。Chiras和Manaster(1977)提出用当日同股票的不同期权的隐含波动率加权得到加权隐含波动率的方法取代历史波动率估计法,并用加权隐含波动率信息为期权定价以及构建对冲组合,获得了显著的超额收益,证明利用B-S模型得到的隐含波动率确实包含有效的信息\cite{CHIRAS1978213}。Macbeth和Merville(1979)利用多个股票期权对模型进行了全面的实证检验\cite{Jame-1979}。他们采用回归模型得到预测的隐含波动率后进行定价,发现期权的价值程度(当前价格相对行权价格)对模型与真实价格之间的差异有着系统性影响。实值期权的模型预测价格偏高,而虚值期权的模型预测价格偏低,同时差异也随期限的缩短而减小。
}
\par{Black-Scholes模型设定波动率为市场已知且恒定,这一假设未必与事实相符。Heston(1993)将波动率设定为CIR随机模型形式,并得到这一条件下期权价格的解\cite{10.1093/rfs/6.2.327}。而Savickas则通过对Black-Scholes模型、Heston模型和Weibull模型分别构建delta对冲策略,发现随机波动率模型的收益并无显著提升,同时也呈现出明显的波动率微笑现象\cite{Rober-2005}。}
\par{
    Hull和White(2017)则重新利用Black-Scholes模型达到了更好的对冲模型效果,他们考虑了资产价格和波动率的相关性之后,在delta对冲基础之上提出了最优对冲法则\cite{Hull-2017}。其根本思想在于通过对已有对冲误差和delta之间的建模来优化对冲误差。
}