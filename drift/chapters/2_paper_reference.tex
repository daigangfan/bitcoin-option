\chapter{相关研究综述}
\section{期权定价领域研究}
\par{
期权定价问题一直是金融研究中的热点。Black和Scholes提出了经典的Black-Scholes模型\cite{10.2307/1831029},其中包含了一系列关于市场有效性的假设,并且对这一模型在市场上的应用进行了实证检验\cite{J-1972}。他们采用滚动窗口的股价收益波动率方法,对纽交所上部分股票的期权进行定价,同时构建对冲策略,证实利用模型信息买入定价低于市场价格、卖出定价高于市场价格的期权同时卖出和买入对应股票的策略可以获得正向的超额收益。同时通过对不同波动率组别的收益进行分析,发现了对高波动率的股票模型会高估期权的价格。}
\par{
之后有众多研究对Black-Scholes模型进行了检验并作出改进。Chiras和Manaster在1977年的研究中,提出用当日同股票的不同期权的隐含波动率加权得到加权隐含波动率的方法取代历史波动率估计法,并用加权隐含波动率信息为期权定价以及构建对冲组合,获得了显著的超额收益,证明利用B-S模型得到的隐含波动率确实包含有效的信息\cite{CHIRAS1978213}。Macbeth和Merville在1979年的文章中利用多个股票期权对模型进行了全面的实证检验\cite{Jame-1979}。他们采用回归模型得到预测的隐含波动率后进行定价,发现期权的价值程度(当前价格相对行权价格)对模型与真实价格之间的差异有着系统性影响。实值期权的模型预测价格偏高,而虚值期权的模型预测价格偏低,同时差异也随期限的缩短而减小。
}
