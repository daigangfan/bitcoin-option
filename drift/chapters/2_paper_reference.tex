\chapter{相关研究综述}
\section{期权定价领域研究}
\par{
期权定价问题一直是金融研究中的热点。Black和Scholes(1972)提出了经典的Black-Scholes模型\cite{10.2307/1831029},其中包含了一系列关于市场有效性的假设,并且对这一模型在市场上的应用进行了实证检验\cite{J-1972}。他们采用滚动窗口的股价收益波动率方法,对纽交所上部分股票的期权进行定价,同时构建对冲策略,证实利用模型信息买入定价低于市场价格、卖出定价高于市场价格的期权同时卖出和买入对应股票的策略可以获得正向的超额收益。同时通过对不同波动率组别的收益进行分析,发现了对于具有较高波动率的股票,定价模型会高估期权的价格。同时他们也验证了套利策略中包含的交易成本是非常巨大的,可能会阻碍套利的进行。}
\par{
之后有众多研究对Black-Scholes模型进行检验或者作出改进。Chiras和Manaster(1977)提出用当日同股票的不同期权的隐含波动率加权得到加权隐含波动率的方法取代历史波动率估计法,并用加权隐含波动率信息为期权定价以及构建对冲组合,获得了显著的超额收益,证明利用B-S模型得到的隐含波动率确实包含有效的信息\cite{CHIRAS1978213}。Macbeth和Merville(1979)利用多个股票期权对模型进行了全面的实证检验\cite{Jame-1979}。他们采用回归模型得到预测的隐含波动率后进行定价,发现期权的价值程度(当前价格相对行权价格)对模型与真实价格之间的差异有着系统性影响。实值期权的模型预测价格偏高,而虚值期权的模型预测价格偏低,同时差异也随期限的缩短而减小。
}
\par{Black-Scholes模型设定波动率为市场已知且恒定,这一假设未必与事实相符。Heston(1993)将波动率设定为CIR随机模型形式,并得到这一条件下期权价格的解\cite{10.1093/rfs/6.2.327}。而Savickas则通过对Black-Scholes模型、Heston模型和Weibull模型分别构建delta对冲策略,发现随机波动率模型的收益并无显著提升,同时也呈现出明显的波动率微笑现象\cite{Rober-2005}。}
\par{
    Hull和White(2017)则重新利用Black-Scholes模型达到了更好的对冲模型效果,他们考虑了资产价格和波动率的相关性之后,在delta对冲基础之上提出了最优对冲法则\cite{Hull-2017}。其根本思想在于通过对已有对冲误差和delta之间的建模来优化对冲误差。
}
\section{数字货币市场相关研究}
\par{数字货币市场近年来的不断升温也吸引了不少学术界的眼光。在价格发现方面,Makarov等人于2018年发表的文章探究了比特币和其他数字货币市场上的价格形成与效率问题\cite{Makarov-2018}。他们发现在数字货币市场之间存在着规模巨大、时间持续的套利空间,但同时这一空间主要来自于对各国对真实货币跨国流通的管制,而各个国家内部的市场间套利的空间相对较小。这一研究证明了比特币市场的价格发现机制相对仍不完善,为我们套利策略的构建提供了佐证。
}
\section{流动性相关研究}
\par{流动性是影响市场有效性的重要因素之一。Shiller(1984)最早提出了市场上存在着两类投资者——精明投资者和噪声投资者\cite{Rober-1984}。当市场的套利成本为0时,精明投资者能够充分利用市场信息套利,使价格回到有效市场水平;随着套利成本增加,噪声投资者对股价的预期的影响会逐渐增加,导致价格偏离有效市场水平。套利成本涉及交易成本、价格滑点等多种与流动性有关的因素。因此我会加入衡量市场流动性的指标来解释期权真实价格与模型价格之间的差异。我主要选择的指标是Amihud(2002)提出的非流动性指标\cite{Yako-2002}。这一指标通过衡量股价收益率对成交量的敏感性来度量市场的流动性,为目前在全球市场上比较通用的指标,具体公式可见第\ref{research_method}章。}