\chapter{引言}
\section{研究背景}
\par{2008年,中本聪在互联网上发布论文《比特币:一种点对点的现金系统》\cite{Nakamoto_bitcoin:a},随后于2009年成功制造出比特币的第一个区块。比特币是一种建立在区块链技术上的去中心化交易凭证。不同于其他货币依赖中心结算制度,比特币通过分布式账本对每一次交易进行验证。
由于其安全、去中心化的特点,比特币很快吸引了众多投资者,逐渐成为一种重要的金融资产。
目前在世界范围内有数百家比特币交易所,合计交易额接近百亿美元。
}
\par{随着比特币交易市场的成熟,除了直接购买比特币以获得增值的需求之外,投资者对于交易过程中更多样的投机方式(如做空或者波动率交易)、更稳定的风险控制(如套期保值)有了更高的需求。
因此比特币衍生品应运而生。2013年6月,国内第一家比特币期货交易所——796交易所出现。在国际市场上,芝加哥商品交易所(CME)和芝加哥期权交易所(CBOE)于2017年底开始交易受到监管的比特币期货合约。
}
\par{期货合约的收益与未来资产价格完全线性相关,其对冲风险和杠杆交易能力都不能完全满足投资者的需求。随着近年来比特市场的价格和波动都大幅提升,不少交易所尝试将证券市场的期权、互换等衍生品在比特币市场上实现。
其中Deribit(https://www.deribit.com)和LedgerX(https://www.ledgerx.com)为两个交易相对活跃的期权交易所。}
\section{研究成果及意义}
\subsection{研究成果}
期权允许购买者在指定到期日及其之前选择是否以指定价格购买标的资产,是一种非常重要的金融衍生品。
投资者可以利用期权的保护特性进行套期保值,也可以利用其杠杆特性进行投机交易。期权的出现为比特币市场的投资提供了更多样化的选择。
\par{本文针对LedgerX交易所公开的期权交易数据对比特币市场上的期权进行分析。主要得到了以下三个部分的结论:}
\par{第一,通过Black-Scholes模型对市场上的期权进行定价,并得到市场价格与模型价格的偏差,通过基本描述性统计发现,认购期权价格普遍被模型高估,而认沽期权价格普遍被模型低估。时间和价值程度对于价格偏差有明显影响,同时也验证了波动率微笑现象在比特币市场上的存在。}
\par{第二,利用线性回归模型,使用和市场特征、期权特性有关的多个变量解释了期权市场价格和模型价格之间的偏差。期权的波动率溢价、delta值、较低的流动性和比特币市场收益率与交易量对于价格偏差有显著的提升作用,即推动市场价格高于模型估计;而到期时间、看涨期权的价值程度(比特币价格/行权价)则对价格偏差有降低作用。我们从期权市场上的供给和需求、以及期权套利的可行性对这些现象做出了解释。}
\par{第三,以定价模型为基础,按照delta-对冲的思路构建套利组合,在多种套利组合上,认购期权获得了显著为正的收益,而认沽期权并没有。我也尝试结合回归结果对这一现象做出了解释。}
\subsection{研究意义}
\par{这一研究具有如下的意义:}
\par{
  第一,比特币市场以及其期权市场出现时间都较晚,本研究较为及时地利用最新数据进行了分析,证明了经典的Black-Scholes模型具有应用在这类新兴市场金融衍生品的价值。同时实证分析选取了一些具有明确意义的变量,比较全面地探究了影响期权价格的因素。
}
\par{
  第二,本研究对于金融投资具有实践意义。期权是金融市场上重要的套期保值和投机套利工具。本研究构建了相对合理的套利组合,是一种重要的期权投资和对冲手段,为投资者在数字货币市场上的投资提供更多样化的选择。同时,对现有欧式期权的研究也为金融机构在数字货币市场上推出更复杂的衍生品提供了参考。
}
\par{本文接下来几个部分包括:第2章将介绍比特币市场和期权定价领域相关研究;第3章将介绍数据来源并描述数据;第4章将展示定价模型、定价结果以及对定价偏差的解释;第5章将展示对冲策略结果;第6章为结论。}