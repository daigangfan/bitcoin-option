\chapter{引言}
\section{研究背景}
\par{2008年,中本聪在互联网上发布论文《比特币:一种点对点的现金系统》\cite{Nakamoto_bitcoin:a},随后于2009年成功制造出比特币的第一个区块。
由于其安全、去中心化的特点,比特币很快吸引了众多投资者,逐渐成为一种重要的金融资产。
目前在世界范围内有数百家比特币交易所,合计交易额接近百亿美元。
}
\par{随着比特币交易市场的成熟,除了直接购买比特币以获得增值的需求之外,投资者对于交易过程中更多样的投机方式(如做空或者波动率交易)、更稳定的风险控制有了更高的需求。
因此比特币衍生品应运而生。2013年6月,国内第一家比特币期货交易所——796交易所出现。在国际市场上,芝加哥商品交易所(CME)和芝加哥期权交易所(CBOE)于2017年底开始交易受到监管的比特币期货合约。
}
\par{期货合约的收益与未来资产价格完全线性相关,其对冲风险和杠杆交易能力都不能完全满足投资者的需求。随着近年来比特市场的价格和波动都大幅提升,不少交易所尝试将证券市场的期权、互换等衍生品在比特币市场上实现。
其中Deribit(https://www.deribit.com)和LedgerX(https://www.ledgerx.com)为两个交易相对活跃的期权交易所。本文将针对LedgerX交易所公开的期权交易数据对比特币市场上的期权特性进行分析。}
\section{研究意义}
期权允许购买者在指定到期日及其之前选择是否以指定价格购买标的资产,是一种非常重要的金融衍生品。
投资者可以利用期权的保护特性进行套期保值,也可以利用其杠杆特性进行投机交易。期权的出现为比特币市场的投资提供了更多样化的选择。
本文对Black-Scholes等期权定价模型在比特币期权市场上的应用进行分析,具有如下的意义:
\begin{itemize}
    \item 期权真实价格和理论价格之间的差异来自市场的无效性以及设定的错误,可以对比特币市场特征有进一步认识。
    \item  比特币市场为少数最近出现且成长较快的市场,这一研究将能够为研究新兴市场上的金融衍生品特征提供参考。
    \item 目前流通期权几乎均为欧式期权,这一研究能够为之后推出更复杂的奇异期权提供参考。
    \item 期权作为一种重要的套期保值和投机套利工具,对其特性进行充分研究有助于对实际投资进行指导,为数字货币市场的投资提供多样化选择。
  \end{itemize}
\section{相关研究成果}  
\subsection{已有比特币研究}

