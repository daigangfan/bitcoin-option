\chapter{结论}
\section{研究结果}
本文通过对比特币期权数据进行分析、定价以及对价差的解释,以Black-Scholes定价模型为定价基准,发现了一些能够解释真实价格偏离模型价格因素的变量,并分析了他们系数的方向及意义。
    \par{第一,对于一个新出现的衍生品市场,供给和需求的不平衡对价格有很大的冲击作用,故市场热度如收益率、交易量等变量对期权价格有正向的提升作用,这一作用主要体现在认沽期权的价格上,市场收益率对认购期权的价格影响则恰恰相反。收益率峰度等对风险的高阶衡量也能够提升对期权的需求进而提升价格。同时,流动性一定程度上衡量了完成期权套利策略的可行性,阻碍了价格回到正常水平。期权投资者掌握的波动率信息和期权所在的价值水平同样对价格的偏离有系统性影响。需要注意的是,这些信息只对价值程度比较接近实值及实值期权有意义,对于价值程度更虚的期权,本研究发现其存在极度不符合模型价值的价格,可能存在一定的炒作现象。}
    \par{第二,构建的策略则体现出认购期权的定价模型具有一定的信息价值,相对来说认购期权受收益率这种比较对称的变量影响较大,故在长期套利策略上有较好的表现,相比之下,认沽期权的定价偏差更多来自异质性因素,来自模型无法解释的部分,故定价模型在认沽期权上的套利策略中包含的信息量不足。总体上,期权的定价偏差理论上能够被用于构建对冲组合进行套利。}
\par{本研究分析了比特币市场与股票市场的相同与不同之处,流动性、价值程度等对定价偏差的影响方向相同,而时间、收益率等因素为比特币市场上具有特殊解释的变量,体现了作为新兴市场较容易受到市场热度影响的特征。为进一步研究比特币市场的有效性、适合比特币期权的定价模型等打下了基础。同时,也为投资实践提供了参考,投资者可以结合其所在市场环境、所需要的期权的特征,及不同特征对期权价格的影响方向,综合考虑定价模型是否会得到偏高或偏低的价格,完善期权投资投资策略。}
\section{研究不足及展望} 
本研究存在一定的不足之处,同时研究中得到的一些结论也为进一步解释这些问题提供了启发:
\begin{itemize}
    \item 第一,定价模型应用较为单一。考虑到比特币市场的多变性,同时之前有部分研究证明其他模型的定价效果未必好于传统B-S模型,此处未选择其他模型,而是用B-S模型中的信息作为一个基准进行研究。其他模型可能追求对现实价格的充分拟合,而忽视了市场价格非均衡价格、受到部分变量系统性影响等因素,从而未必在套利策略部分表现好于B-S模型,这一点也可以被最后采用的“最优对冲策略”支持。后续可以考虑对波动率建模的GARCH模型,这一模型与市场收益率的峰度有关,一定程度上会弥补B-S模型假设的不足。
    \item 第二,构建套利策略过程中未考虑交易成本。交易成本是影响期权对冲和套利限制的重要因素,但期权的成本数据未公开,比特币目前的交易成本并不统一,此处难以获取可信的数据。可以在有确定的交易成本数据之后将其纳入收益计算中。
    \item 第三,期权自身流动性因素考虑不足。从公开数据来看,期权自身交易量普遍不高,可能存在流动性的问题,但由于LedgerX本身既是交易所也是清算机构,所以具有和买卖双方结算的能力,故可以认为可以以交易所为对手方开出一支期权的买卖价格,期权的流动性影响可能不大。对回归模型中期权持仓量的影响分析、逐日和连续策略的结果类似也均能够支持期权流动性对模型定价效力影响不大的结论。
    \item 第四,未能进行子时段分析。考虑到目前可获取数据较少,故没有考虑不同时段的分析,可以在未来期权进一步发展具有更充足数据的情况下进行更充分分析。
\end{itemize}