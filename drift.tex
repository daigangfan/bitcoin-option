
\documentclass[degree=bachelor]{thuthesis}
\usepackage{thuthesis}
\usepackage{booktabs}

\begin{document}
\chapter{研究方法}
    \section{期权的定价}
    本文首先采用Black-Scholes模型\cite{10.2307/1831029}对比特币进行定价。Black-Scholes模型对交易资产及其市场有如下假设:
    \begin{itemize}
        \item 存在已知且恒定的无风险收益率r。
        \item 资产价格为带漂移随机游走过程, 即$dS={\mu}Sdt+{\sigma}SdW$,其中价格波动率$\sigma$为已知的常数。可以推导出在这一假设下资产的收益率服从对数正态分布。
        \item 期权为欧式期权,即只有到期时期权的购买者才能够行权。
        \item 买卖期权和标的资产时不会产生交易费用和其他损失。
        \item 能够以无风险利率自由借入借出任意数量的资产。
        \item 对卖空操作无限制。
    \end{itemize}
    在以上假设下,市场上不存在套利机会。

\chapter{描述性统计}

\begin{table}[!ht]
    \centering
    \caption{比特币数据描述性统计}
    \begin{tabular}{lrrrrr}
\toprule
{} &             成交量 &   对数收益率 &     波动率 &      偏度 &      峰度 \\
\midrule
count &         544.000 & 544.000 & 544.000 & 544.000 & 544.000 \\
mean  &  6749637776.700 &  -0.000 &   0.047 &  -0.096 &   2.667 \\
std   &  3790972106.645 &   0.044 &   0.007 &   0.189 &   0.802 \\
min   &  1403920000.000 &  -0.185 &   0.032 &  -0.541 &   1.656 \\
25\%   &  4273417520.000 &  -0.017 &   0.043 &  -0.313 &   2.135 \\
50\%   &  5470000143.500 &   0.002 &   0.049 &  -0.041 &   2.326 \\
75\%   &  7810741265.500 &   0.018 &   0.054 &   0.060 &   3.032 \\
max   & 23840899072.000 &   0.225 &   0.055 &   0.175 &   5.026 \\
\bottomrule
\end{tabular}

\end{table}

\begin{table}[!ht]
    \centering
    \caption{期权数据描述性统计}
\begin{tabular}{lrrrr}
\toprule
{} &      期限 &       行权价 &  认购期权数量 &  认沽期权数量 \\
\midrule
count & 306.000 &   306.000 & 197.000 & 109.000 \\
mean  &  90.536 &  8867.647 &         &         \\
std   & 141.132 &  6207.192 &         &         \\
min   &   1.000 &  2000.000 &         &         \\
25\%   &   3.250 &  6000.000 &         &         \\
50\%   &  30.000 &  7000.000 &         &         \\
75\%   &  96.750 & 10000.000 &         &         \\
max   & 637.000 & 50000.000 &         &         \\
\bottomrule
\end{tabular}

\end{table}

\begin{table}[!ht]
    \centering
    \caption{期权定价差异分组统计}
    \begin{tabular}{lrrrrr}
\toprule
time\_cut &  0, 20 &  20, 80 &  80, 180 &  180, 637 &   mean \\
moneyness\_cut &          &           &            &             &        \\
\midrule
0.0, 0.6    &    24.78 &     49.11 &      90.46 &      100.22 &  86.70 \\
0.6, 0.8    &    48.66 &     41.92 &     -35.64 &     -104.51 & -18.29 \\
0.8, 1.2    &    -5.37 &     29.51 &      97.81 &     -215.37 &   3.19 \\
1.2, 3.8    &    -9.18 &    -12.77 &      93.59 &       72.38 &  27.93 \\
mean          &     0.68 &     26.70 &      68.22 &       -9.34 &        \\
\bottomrule
\end{tabular}

\end{table}

\begin{table}[!ht]
    \centering
    \caption{期权定价差异绝对值分组统计}
    \begin{tabular}{lrrrrr}
\toprule
time\_cut &  0, 20 &  20, 80 &  80, 180 &  180, 637 &   mean \\
moneyness\_cut &          &           &            &             &        \\
\midrule
0.0, 0.6    &    37.39 &     67.10 &     215.39 &      465.69 & 309.80 \\
0.6, 0.9    &    80.84 &    141.96 &     302.62 &      583.87 & 269.25 \\
0.9, 1.1    &    85.07 &    132.13 &     294.32 &      591.21 & 149.56 \\
1.1, 3.8    &   224.98 &    196.33 &     361.22 &      358.52 & 264.30 \\
mean          &   107.93 &    143.37 &     292.74 &      490.58 &        \\
\bottomrule
\end{tabular}

\end{table}

\chapter{回归分析}

\begin{table}[!ht]
    \centering
    \caption{线性回归模型结果}
    \begin{tabular}{lc}
\hline
                   &    定价偏差     \\
\midrule
\midrule
Intercept          & -5.8095***  \\
                   & (1.2262)    \\
log\_ret           & 0.4303      \\
                   & (0.3276)    \\
kurtosis           & 0.1946***   \\
                   & (0.0290)    \\
amihud             & 4.1291***   \\
                   & (1.2595)    \\
maxmin\_ratio      & 0.7072      \\
                   & (0.8066)    \\
btc\_volume        & 0.2097***   \\
                   & (0.0600)    \\
delta\_5           & 0.0847      \\
                   & (0.1568)    \\
vol\_pre           & 11.2719***  \\
                   & (0.9403)    \\
open\_interest     & -0.0000     \\
                   & (0.0002)    \\
slope              & -0.0008     \\
                   & (0.0315)    \\
contract\_is\_call & 0.7138**    \\
                   & (0.2886)    \\
inter\_call\_money & -0.2749**   \\
                   & (0.1140)    \\
inter\_put\_money  & 0.7044***   \\
                   & (0.2508)    \\
observations       & 577.0000    \\
R-Squared          & 0.5975      \\
Adjusted R-Squared & 0.5889      \\
\hline
\end{tabular}
\end{table}

\bibliographystyle{plain}
\bibliography{paperbib}
\end{document}
