
\documentclass[degree=bachelor]{thuthesis}
\usepackage{thuthesis}
\usepackage{booktabs}

\begin{document}
\chapter{研究方法}
    \section{期权的定价}
    本文首先采用Black-Scholes模型\cite{10.2307/1831029}对比特币进行定价。Black-Scholes模型对交易资产及其市场有如下假设:
    \begin{itemize}
        \item 存在已知且恒定的无风险收益率r。
        \item 资产价格为带漂移随机游走过程, 即$dS={\mu}Sdt+{\sigma}SdW$,其中价格波动率$\sigma$为已知的常数。可以推导出在这一假设下资产的收益率服从对数正态分布。
        \item 期权为欧式期权,即只有到期时期权的购买者才能够行权。
        \item 买卖期权和标的资产时不会产生交易费用和其他损失。
        \item 能够以无风险利率自由借入借出任意数量的资产。
        \item 对卖空操作无限制。
    \end{itemize}
    在以上假设下,市场上不存在套利机会。

\chapter{描述性统计}

\begin{table}[!ht]
    \centering
    \caption{比特币数据描述性统计}
    \begin{tabular}{lrrrrr}
\toprule
{} &             成交量 &   对数收益率 &     波动率 &      偏度 &      峰度 \\
\midrule
count &         545.000 & 545.000 & 545.000 & 545.000 & 545.000 \\
mean  &  6756320922.978 &  -0.000 &   0.047 &  -0.097 &   2.669 \\
std   &  3790698285.665 &   0.044 &   0.007 &   0.188 &   0.804 \\
min   &  1403920000.000 &  -0.185 &   0.032 &  -0.541 &   1.656 \\
25\%   &  4273640000.000 &  -0.017 &   0.043 &  -0.313 &   2.135 \\
50\%   &  5475579904.000 &   0.002 &   0.049 &  -0.042 &   2.327 \\
75\%   &  7826525254.000 &   0.018 &   0.054 &   0.060 &   3.041 \\
max   & 23840899072.000 &   0.225 &   0.055 &   0.175 &   5.026 \\
\bottomrule
\end{tabular}

\end{table}

\begin{table}[!ht]
    \centering
    \caption{期权数据描述性统计}
\begin{tabular}{lrrrr}
\toprule
{} &      期限 &       行权价 &  认购期权数量 &  认沽期权数量 \\
\midrule
数目  & 306.000 &   306.000 & 197.000 & 109.000 \\
均值  &  90.536 &  8867.647 &         &         \\
标准差 & 141.132 &  6207.192 &         &         \\
最小值 &   1.000 &  2000.000 &         &         \\
中位数 &  30.000 &  7000.000 &         &         \\
最大值 & 637.000 & 50000.000 &         &         \\
\bottomrule
\end{tabular}

\end{table}

\begin{table}[!ht]
    \centering
    \caption{期权定价差异分组统计}
    \begin{tabular}{lrrrrr}
\toprule
time\_cut &  0, 20 &  20, 80 &  80, 180 &  180, 637 &   mean \\
moneyness\_cut &          &           &            &             &        \\
\midrule
0.0, 0.6    &    27.25 &     65.22 &     154.80 &      271.67 & 191.15 \\
0.6, 0.9    &    46.44 &     64.38 &      52.10 &      293.21 & 114.55 \\
0.9, 1.1    &   -47.41 &    -90.56 &     561.17 &      819.41 &  31.52 \\
1.1, 3.8    &    48.47 &     80.85 &     602.66 &      729.05 & 299.03 \\
mean          &    -6.89 &      7.56 &     284.52 &      426.61 &        \\
\bottomrule
\end{tabular}

\end{table}

\begin{table}[!ht]
    \centering
    \caption{期权定价差异绝对值分组统计}
    \begin{tabular}{lrrrrr}
\toprule
time\_cut &  0, 20 &  20, 80 &  80, 180 &  180, 637 &   mean \\
moneyness\_cut &          &           &            &             &        \\
\midrule
0.0, 0.6    &    37.39 &     67.10 &     215.39 &      465.69 & 309.80 \\
0.6, 0.8    &    75.77 &    121.04 &     349.13 &      653.78 & 327.03 \\
0.8, 1.2    &    94.57 &    149.22 &     306.75 &      484.31 & 173.94 \\
1.2, 3.8    &   276.90 &    191.20 &     299.21 &      323.15 & 259.23 \\
mean          &   107.93 &    143.37 &     292.74 &      490.58 &        \\
\bottomrule
\end{tabular}

\end{table}

\chapter{回归分析}

\begin{table}[!ht]
    \centering
    \caption{线性回归模型结果}
    \begin{tabular}{lc}
\hline
                   &    ����ƫ��     \\
\midrule
\midrule
Intercept          & -8.9503***  \\
                   & (1.3354)    \\
log\_ret           & 1.0493***   \\
                   & (0.3545)    \\
amihud             & 4.5516***   \\
                   & (1.3901)    \\
maxmin\_ratio      & 4.2845***   \\
                   & (0.8111)    \\
btc\_volume        & 0.2195***   \\
                   & (0.0663)    \\
delta\_5           & 0.3884**    \\
                   & (0.1592)    \\
vol\_pre           & 8.0500***   \\
                   & (1.0823)    \\
spread             & 0.0001      \\
                   & (0.0001)    \\
open\_interest     & -0.0001     \\
                   & (0.0002)    \\
slope              & -87.0439    \\
                   & (84.7885)   \\
contract\_is\_call & 0.3161      \\
                   & (0.3113)    \\
inter\_call\_money & -0.3855**   \\
                   & (0.1822)    \\
inter\_put\_money  & 0.4704**    \\
                   & (0.2313)    \\
observations       & 577.0000    \\
R-Squared          & 0.5143      \\
Adjusted R-Squared & 0.5040      \\
\hline
\end{tabular}
\end{table}

\bibliographystyle{plain}
\bibliography{paperbib}
\end{document}
